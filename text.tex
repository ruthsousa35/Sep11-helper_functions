$
Javascript Variables and Functions

//let myVar = "win" Javascript

% html
<!- comments->
_____________________________

Variable - storage value - can be modified 

\begin{definition}[name of the definition]
    
\end{definition}

Naming variables 


_____________________________
 
Statement block { }
code that you will execute

______________________________

var myVar = "Wandinha";

// camelCase

//snake_camelCase


___________________________

let MyVar ="John";
console.log(MyVar); /// define first

_____________________________________________________________________________________________

var = its global,  case sensitive, not very good, scope=rule = when you need everywhere

_____________________________________________________________________________________________


Understanding Hoisting 

( the order matters)

console.log(myVar);
//unddefined
var myVar = "Wandinha";
_____________________________________________________________________________________________


let = can be change the value

{ let myVar ="Wandinha";  }
_____________________________________________________________________________________________
const = don't change ( block-scoped - not allow reassignment, but array can be modified)

{
    const myVar ="John";
    console.log(myVar);
}

console.log(myVar);

_____________________________________________________________________________________________

Type Errors

typeof - operator   
_____________________________________________________________________________________________

Function - reusable block of code
( don't use num or name with name)
 Call and invoke

function sayHello() {
    console.log("Hello, world!");
}

sayHello();

 function addTwo(num) {
    num = 999; 
    return num +2;

 }

 addTwo(3);
 addTwo(num);
 addTwo(num3);

__________________________
 command I - called AI

_____________________________________________________________________________________________


% Named Function Example

\begin{verbatim}
function multiply(a, b) {
    return a + b;
}

let result = add(2, 3);
console.log(result); // Output: 5
\end{verbatim}
_________________________

Statement and expression
A statement in JavaScript is a complete instruction that performs an action, such as declaring a variable or calling a function. For example:
let x = 5;

An expression is any valid unit of code that resolves to a value. For example:
2 + 3
x * 10

In functions, you can have both statements and expressions. For example, the body of a function can contain statements (like variable declarations) and expressions (like calculations or return values).

_____________________________________________

% Anonymous Function Example

\begin{verbatim}
let greet = function(name) {
    console.log("Hello, " + name + "!");
};

greet("Alice"); // Output: Hello, Alice!
\end{verbatim}

(callback = diferent ways )

_____________________________________________

% Immediately Invoked Function Expression (IIFE) Example

\begin{verbatim}
(function() {
    console.log("This function runs immediately!");
})();
\end{verbatim}

_____________________________________________


% Arrow Function Example

\begin{verbatim}
const add = (a, b) => {
    return a + b;
};

console.log(add(2, 3)); // Output: 5
\end{verbatim}

_____________________________________________

function actually object

this 

is a keyword

% Arrow functions do not have their own 'this'
% Example:

\begin{verbatim}
const obj = {
    value: 42,
    regularFunc: function() {
        console.log(this.value); // 'this' refers to obj
    },
    arrowFunc: () => {
        console.log(this.value); // 'this' is not bound to obj
    }
};

obj.regularFunc(); // Output: 42
obj.arrowFunc();   // Output: undefined (or window/global value)
\end{verbatim}

% In arrow functions, 'this' is inherited from the surrounding scope.

usually is a const

_____________________________________________

variable that i set to function:


const subtract = (x,y) => x-y;

const subtract = (x,y) =>{
   return x-y;
};





const subtract = (x,y) => x-y;

_____________________________________________


function subtract(x,y) { x-y;

    this 
    return x-y;
}

_____________________________________________

callback = Arrow


object programming - arrow

_____________________________________________


% Object Example

\begin{verbatim}
const person = {
    name: "Alice",
    age: 30,
    greet: function() {
        console.log("Hello, my name is " + this.name);
    }
};

console.log(person.name); // Output: Alice
person.greet(); // Output: Hello, my name is Alice
\end{verbatim}

_____________________________________________

MDM - Javascript
documentation 

closure
_____________________________________________


call Function
Invoking Function 

% Example of calling a function

\begin{verbatim}
function sayHello() {
    console.log("Hello!");
}

// Call (invoke) the function
sayHello(); // Output: Hello!
\end{verbatim}


_____________________________________________


% Callback Function Example

\begin{verbatim}
function fetchData(callback) {
    setTimeout(function() {
        const data = "Sample data";
        callback(data);
    }, 1000);
}

function handleData(result) {
    console.log("Received:", result);
}

fetchData(handleData); // Output after 1 second: Received: Sample data
\end{verbatim}



__________________________________________

% Event Listener Example

lovable.dev. clone website


_____________


\begin{verbatim}
// HTML: <button id="myButton">Click me</button>

document.getElementById("myButton").addEventListener("click", function() {
    console.log("Button was clicked!");
});
\end{verbatim}

__________________________________________
% Example: Button that changes text when clicked

\begin{verbatim}
// HTML: <button id="changeBtn">Change Text</button>
//       <p id="text">Original Text</p>

document.getElementById("changeBtn").addEventListener("click", function() {
    document.getElementById("text").textContent = "Text changed!";
});
\end{verbatim}
__________________________________________

% Example: Using addEventListener to handle a click event

\begin{verbatim}
// HTML: <button id="alertBtn">Show Alert</button>

document.getElementById("alertBtn").addEventListener("click", function() {
    alert("Button was clicked!");
});
\end{verbatim}

__________________________________________
